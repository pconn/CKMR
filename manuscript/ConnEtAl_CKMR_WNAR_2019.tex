\documentclass[serif,mathserif]{beamer}
\usepackage{amsmath, amsfonts, epsfig, xspace, color, colortbl,natbib,xcolor}
\usepackage{algorithm,algorithmic}
\usepackage{animate}
%\usepackage{pstricks,pst-node}
%\usepackage{multimedia}
%\usepackage{movie15}
\usepackage[normal,tight,center]{subfigure}
\setlength{\subfigcapskip}{-.5em}
%\usepackage{beamerthemesplit}
%\usetheme{lankton-keynote}

%%%
% PRELIMINARY FORMATTING
%%%
\renewcommand\sfdefault{phv}
\renewcommand\familydefault{\sfdefault}
\usetheme{default}
\usepackage{color}
\useoutertheme{default}
\usepackage{texnansi}
\usepackage{marvosym}
\definecolor{bottomcolour}{rgb}{0.32,0.3,0.38}
\definecolor{middlecolour}{rgb}{0.08,0.08,0.16}
\definecolor{noaaturq}{RGB}{2,171,216}
\definecolor{noaablue}{RGB}{20,21,127}
\setbeamerfont{title}{size=\huge}
\setbeamercolor{structure}{fg=gray}
\setbeamertemplate{frametitle}[default]%[center]
\setbeamercolor{normal text}{bg=black, fg=white}
\setbeamercolor{block title}{parent=normal text, bg=noaaturq}%noaablue!60!white}
\setbeamercolor{block body}{parent=normal text, use=block title,bg=noaablue}%block title.bg!50!black}
\setbeamertemplate{background canvas}[vertical shading]
[bottom=bottomcolour, middle=middlecolour, top=black]
\setbeamertemplate{items}[circle]
\setbeamerfont{frametitle}{size=\huge}
\setbeamertemplate{navigation symbols}{} %no nav symbols
\setbeamertemplate{blocks}[rounded]
\graphicspath{{c:/users/paul.conn/git/STabundance/ISEC_presentation}}

%\usecolortheme{seahorse}


\begin{document}

\frame{
\centering
\renewcommand{\baselinestretch}{1.8}\normalsize
{\LARGE \textcolor{noaaturq}{Preferential sampling in species distribution models}}\\
\bigskip\bigskip
\renewcommand{\baselinestretch}{1.25}\normalsize
Paul B. Conn\\
\footnotesize \textcolor{lightgray}{\em Marine Mammal Lab, NOAA Alaska Fisheries Science Center, Seattle, WA }\\
\textcolor{lightgray}{Email: paul.conn@noaa.gov}\\
\bigskip\bigskip
\textcolor{lightgray}{PSAW II\\
La Jolla, San Diego\\
February 12, 2019}\\	
\vspace*{\fill}
\begin{figure}
%	%\subfigure{\includegraphics[height=2cm]{UAFlogo.png}}
%	\hspace*{\fill}
%	\subfigure{\includegraphics[height=2cm]{noaa_logo.png}}
\hspace{\fill}
\includegraphics[height=1.5cm]{NOAA-logo.pdf}
\end{figure}
}

\section{Background} % add these to see outline in slides
\begin{frame}
\frametitle{Outline}
  \begin{enumerate}
    \item Definitions
    \item Models
    \item Simulations
    \item Seals
    \item Birds
    \item Final thoughts
  \end{enumerate}
\end{frame}



\section{Background} % add these to see outline in slides
\begin{frame}
\frametitle{Definitions}
  \textcolor{noaaturq}{\textbf{What is preferential sampling?}} \\ \pause

  When a spatial process of interest and the locations chosen for sampling are conditionally dependent \textit{given} modeled covariates. \pause

\vspace{0.5cm}

\textcolor{noaaturq}{Data} (counts, presence/absence): ${\bf Y} = \{ Y_1, Y_2, \hdots, Y_n \}$ \pause

\vspace{0.25cm}

\textcolor{noaaturq}{Latent spatial process} (abundance, occupancy): ${\bf Z} = \{ Z_1, Z_2, \hdots, Z_S \}$ \pause

\vspace{0.25cm}

\textcolor{noaaturq}{Covariates}: ${\bf X} = \{ {\bf x}_1, {\bf x}_2, \hdots, {\bf x}_S \}$ \pause

\vspace{0.25cm}

\textcolor{noaaturq}{Sample inclusion indicator}: ${\bf R} = \{ R_1, R_2, \hdots, R_S \}$ \pause

\begin{block}{Definition: Preferential sampling (after \citet{DiggleEtAl2010}, \citet{PatiEtAl2011})}
      Preferential sampling exists if $[{\bf R},{\bf Z} | {\bf X}] \neq [{\bf R}|{\bf X}][{\bf Z}|{\bf X}]$
\end{block}
\end{frame}



\begin{frame}
  \frametitle{Definitions: pop quiz!}
  \vspace{0.75cm}

  \textcolor{noaaturq}{Example 1} An investigator (or volunteer) uses prior knowledge of animal density to select locations for sampling. \\ \pause

  \vspace{0.5cm}

  \textcolor{noaaturq}{Example 2} An investigator is more likely to sample sites close to a base of operations, and animal densities there are higher (or lower) than expected based on explanatory covariates alone.

\end{frame}


\begin{frame}
  \frametitle{Models}
  \vspace{0.75cm}

  Most published work on preferential sampling in continuous spatial domain w/ Gaussian processes. \pause

 \vspace{0.25cm}
  Here, I consider a species distribution model for count data on a discrete spatial domain.\\
  \vspace{0.25cm}
   \textcolor{noaaturq}{Latent abundance }:
   \begin{eqnarray*}
      Z_i  & = & A_i \mu_i. \\
   \log(\mu_i)  & = & \beta_0 + {\bf x}_i \boldsymbol{\beta}+\delta_i \\
   \boldsymbol{\delta} & \sim & \textrm{MVN}(0,\boldsymbol{\Sigma})
   \end{eqnarray*} \pause
   \textcolor{noaaturq}{Observation model }:
   \begin{eqnarray*}
      Y_i  & \sim & \textrm{Poisson}(Z_i p_i)
   \end{eqnarray*}
\end{frame}



\begin{frame}
\frametitle{Preferential sampling model}
  \textcolor{noaaturq}{Sample inclusion indicator}:
    \begin{eqnarray*}
      R_i & \sim & \textrm{Bernoulli}(\nu_i)\\
    \textrm{logit}(\nu) & = & \beta_0^* + {\bf X}^* \boldsymbol{\beta}^*+\boldsymbol{\eta}+{\bf B} \boldsymbol{\delta} \\
      \boldsymbol{\eta} & \sim & \textrm{MVN}(0,\boldsymbol{\Sigma}^*)
    \end{eqnarray*} \pause

    \vspace{0.25cm}

    Note that $\boldsymbol{\delta}$ random effects are also included in the latent abundance model, providing for possible dependence. \pause

    \vspace{0.5cm}

    The matrix ${\bf B}$ describes preferential sampling.  We set  ${\bf B} = b{\bf I}$ (if one is brave, more sophistication possible). \pause

     \vspace{0.5cm}

    \textcolor{noaaturq}{Total abundance posterior prediction}:
    $\tilde{N} = \sum_i Z_i$

\end{frame}

\begin{frame}
\frametitle{Simulations}
  \textcolor{noaaturq}{Rationale}:
  \begin{enumerate}
  \item See what effects preferential sampling
  has on abundance estimation
  \item See how well our modeling framework does in alleviating bias
  \end{enumerate} \pause
  \begin{columns}[c]
\column{2in}
  Three simulation scenarios:
  \begin{itemize}
    \item No preferential sampling,
    \item moderate preferential sampling,
    \item pathological preferential sampling
   \pause
   \end{itemize}\column{2in}
\framebox{\includegraphics[width=2in]{sim_R.pdf}}
\end{columns}
\end{frame}

\begin{frame}
\frametitle{Simulations}
  \begin{columns}[c]
\column{1.5in}
\begin{itemize}
\item Matern covariance structure
\item Laplace approximation via Template Model Builder \citep{KristensenEtAl2016}
\item 500 simulations per design point
\item Details in \cite{ConnEtAl2017}
\end{itemize}
\column{2.8in}
\framebox{\includegraphics[width=2.8in]{Pref_samp_sim_maps.pdf}}
\end{columns}
\end{frame}


\begin{frame}
\frametitle{Simulations}
\framebox{\includegraphics[width=3in]{bias.pdf}}
\end{frame}




\begin{frame}
\frametitle{Seals}
Ice-associated seal aerial surveys - Bering Sea 2012
\begin{columns}[c]
\column{2in}
\framebox{\includegraphics[width=2in]{Scenario_H5.pdf}} \pause
\column{2in}
\framebox{\includegraphics[width=2in]{BOSS2012_Effort.jpg}} \pause
\end{columns}
\vspace{.25in}
 \textcolor{noaaturq}{Preferential sampling?}
%\begin{columns}[c]
%\column{2in}
%\framebox{\includegraphics[width=2in]{ice_april1.pdf}}
%\column{2in}
%\end{columns}
\end{frame}






\begin{frame}
\frametitle{Seals}
  \begin{columns}[c]
\column{2in}
\begin{itemize}
\item Limit to aerial survey counts obtained in one week period (April 10-16, 2012)
\item Ignore multiple detection process vagaries
\item Maximum coverage of strip transects in each 625km$^2$ cell is 3.3\%
\end{itemize}
\column{2.5in}
\framebox{\includegraphics[width=2.5in]{Bearded_counts.pdf}} \pause
\end{columns}
\end{frame}

\begin{frame}
\frametitle{Seals}
  \begin{columns}[c]
\column{1.5in}
\framebox{\includegraphics[width=1.5in]{bearded_pic.jpg}} \pause
\column{3in}
\begin{itemize}
\item 4 models fit to bearded seal counts \pause
\item Covariates: yes/no \pause
\item Preferential sampling parameter $b$ estimated: yes/no \pause
\item Covariates modeled:
    \begin{itemize}
     \item \textcolor{noaaturq}{sea ice concentration} (linear + quadratic)
     \item \textcolor{noaaturq}{distance from southern ice edge} (linear) \pause
     \end{itemize}
\item Same TMB estimation procedure as in simulations
\end{itemize}
\end{columns}
\end{frame}

\begin{frame}
\frametitle{Seals}
\begin{tabular}{lccccc}
  \hline
  Model & LogL & k & CV-MSPE (SE)& $\Delta \textrm{AIC}$ & $\hat{N}$(SE) \\
  \hline
 $M_{cov=0,b=0}$ & -852.7 & 5 & 87.2 & 99.1 & 70738 (6988) \\
 $M_{cov=1,b=0}$ & -847.6 & 12 & 88.3 & 103.0 & 66989 (20374) \\
 $M_{cov=1,b=1}$ & -795.1 & 13 & 105.3 & 0.0 & 40656 (3664) \\
 $M_{cov=0,b=1}$ & -803.0 & 6 & 116.2 & 1.9 & 43232 (2778) \\
  \hline
\end{tabular} \pause

\vspace{.3cm}

Comments:
\begin{itemize}
  \item Difference in support based on model selection criterion  \pause
  \item Model with $b$ estimated had $\hat{b} > 9.0$! \pause
  \item Conclusion: preferential sampling model unstable; use model w/ $b=0$
\end{itemize}
\end{frame}

\begin{frame}
\frametitle{Birds}
Peregrine falcon presence-absence surveys at sites in the French Jura Alps

\vspace{.5cm}

\colorbox{white}{\includegraphics[width=3.2in]{peregrine_presence.png}}
\end{frame}

\begin{frame}
\frametitle{Birds}
Peregrine estimates

\vspace{.5cm}

\colorbox{white}{\includegraphics[width=3.2in]{peregrine_estimates.png}}
\end{frame}

\begin{frame}
\frametitle{Final thoughts}
  \begin{enumerate}
    \item Model-based inference is not a free pass \pause
    \item It would be really nice not to go to such modeling elaborations: include design-based protocols in design! \pause
    \item Include covariates and stratification where possible \pause
    \item Can use methods such as those here for diagnosis of preferential sampling and possible bias reduction \pause
    \item Would be interesting to see what happens when spatial autocorrelation is represented through first order structure (e.g., GAMs) \pause
    \item Applications in other areas: e.g. CPUE standardization in fisheries, time series
  \end{enumerate}
\end{frame}

 \section{End notes}
  \begin{frame}
  \frametitle{}
  {\Huge Questions?}

  \vspace{0.5cm}

  \textcolor{noaaturq}{Acknowledgments} \\
  \begin{itemize}
  \item Coauthors: Jim Thorson, Devin Johnson.
  \item Bird plots from Marc K$\acute{\textrm{e}}$ry, Swiss Ornithological Institute
  \item Seal field research conducted by PEP program at NOAA/AFSC/MML, funded by NOAA and BOEM.
  \end{itemize}
  \vspace{0.5cm}
   \textcolor{noaaturq}{References}
 \bibliographystyle{plainnat}
 {\footnotesize
  \bibliography{master_bib}}

 \end{frame}

\end{document}
